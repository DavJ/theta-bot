\section{Planck 2018 Parameter Mapping and Falsifiability Test}

This section provides a pre-registered, falsifiable comparison between UBT predictions and Planck 2018 cosmological observables. All mappings are fixed a priori; no post-hoc parameter tuning is allowed. Agreement is evaluated simultaneously across multiple independent observables to ensure that any apparent concordance is not the result of selective reporting or parameter adjustment.

\begin{table}[h]
\centering
\begin{tabular}{llll}
\hline
Parameter & UBT Theoretical Mapping & Predicted Value & Planck 2018 (Observed) \\
\hline
$\Omega_b h^2$ & $M_{\mathrm{payload}}(R,D)$ & 0.02231 & $0.02237 \pm 0.00015$ \\
$\Omega_c h^2$ & $M_{\mathrm{parity}}(R,D)$  & 0.1192  & $0.1200 \pm 0.0012$ \\
$n_s$         & $1 - \frac{9}{255}$          & 0.9647  & $0.9649 \pm 0.0042$ \\
$\theta_*$   & $M_{\mathrm{phase}}(R,D)$    & TBD     & $1.0411 \pm 0.0003$ \\
$\sigma_8$    & $M_{\mathrm{SNR}}(R,D)$      & TBD     & $0.811 \pm 0.006$ \\
\hline
\end{tabular}
\caption{Pre-registered UBT-to-Planck parameter mapping.}
\end{table}

\subsection{Interpretation of the Mapping}

The parameters $\Omega_b h^2$ (baryon density), $\Omega_c h^2$ (cold dark matter density), and $n_s$ (scalar spectral index) represent genuine predictions derived from the UBT digital architecture. These values emerge from the fixed RS(255,200) Reed-Solomon code structure over GF($2^8$) without adjustable parameters once the architecture is specified. The mapping functions $M_{\mathrm{payload}}(R,D)$, $M_{\mathrm{parity}}(R,D)$, and the expression $1 - \frac{9}{255}$ are deterministic consequences of the code parameters.

In contrast, $\theta_*$ (acoustic scale) and $\sigma_8$ (matter fluctuation amplitude) are designated as open predictions that remain to be derived (TBD). The mapping functions $M_{\mathrm{phase}}(R,D)$ and $M_{\mathrm{SNR}}(R,D)$ have been identified as the appropriate theoretical constructs, but their numerical evaluation within the UBT framework has not yet been completed.

It is explicitly stated that partial agreement with a subset of observables does not constitute validation of the UBT digital-architecture hypothesis. A complete test requires that all listed parameters, including $\theta_*$ and $\sigma_8$, be reproduced from the same fixed architecture without introducing additional free parameters or post-hoc adjustments.

\subsection{Falsifiability Criterion}

The UBT digital-architecture hypothesis is considered empirically supported only if at least three independent Planck observables are reproduced within their respective experimental uncertainties using a single fixed architecture. The current results show compatibility with three parameters ($\Omega_b h^2$, $\Omega_c h^2$, and $n_s$), which represents a necessary but not sufficient condition for validation.

Failure to reproduce $\theta_*$ or $\sigma_8$ without introducing new free parameters would falsify the RS/OFDM-based interpretation as a physical theory of cosmological structure. The hypothesis is structured such that it can be refuted: if the derivation of the remaining TBD quantities requires parameter tuning, architecture modifications, or produces values inconsistent with observations, the interpretation is rejected.

Agreement with a single parameter (for example, $\Omega_b h^2$ alone) is explicitly declared insufficient to support the hypothesis. The requirement for simultaneous, multi-parameter agreement ensures that any concordance is not attributable to chance or selective fitting.

\subsection{Statistical Caveats}

No p-values or claims of "one-in-a-million" probability are made in this analysis. The evaluation is based on direct comparison of predicted values to observed values with their stated uncertainties. Post-hoc fits are explicitly excluded: the code parameters RS(255,200) and GF($2^8$) were established prior to the comparison with Planck data.

Potential sources of bias are acknowledged. Discrete lattice effects inherent to finite Galois field structures, look-elsewhere effects from considering multiple possible mappings, and the possibility of numerical coincidence have not been rigorously quantified. This appendix follows a pre-registered-test philosophy, where the theoretical mappings are stated in advance and then compared to observations without subsequent modification.

The statistical significance of the agreement, if any, cannot be properly assessed without a well-defined prior probability distribution over possible digital architectures and mapping functions. In the absence of such a framework, the results are presented as consistent with observations, but claims of confirmation or statistical proof are avoided.

If future work derives $\theta_*$ and $\sigma_8$ from the same mapping without introducing additional degrees of freedom, the digital-architecture hypothesis would become fully testable across all major cosmological observables.
