
\documentclass[12pt]{article}
\usepackage{amsmath,amssymb}
\usepackage{geometry}
\usepackage{physics}
\usepackage{hyperref}
\geometry{margin=1in}

\title{Rozšířený čas v komplexních bikvaternionech a ortogonalita v~$\theta$-bázi}
\author{}
\date{\today}

\begin{document}

\maketitle

\section{Motivace}
Ve standardním rámci prediktivních modelů založených na $\theta$-funkcích se čas obvykle chápe jako komplexní proměnná $\tau = t + i\phi$, kde $\phi$ odpovídá fázovému posunu. Tento přístup však nepostihuje dostatečně vnitřní symetrie a latentní struktury trhu. V rámci \emph{Unified Biquaternion Theory} (UBT) navrhujeme model, kde je čas rozšířen na \emph{komplexní bikvaternion}.

\section{Definice bikvaternionového času}
Bikvaternion je prvek z $\mathbb{B} = \mathbb{H} \otimes \mathbb{C}$, kde $\mathbb{H}$ jsou Hamiltonovy kvaterniony a $\mathbb{C}$ komplexní čísla. Zavedeme čas jako
\begin{equation}
\tau = t + \Phi, \quad \Phi \in \mathbb{B},
\end{equation}
kde $t \in \mathbb{R}$ je fyzikální čas a $\Phi$ je komplexní bikvaternion (latentní složka). Explicitně:
\begin{equation}
\Phi = a + b\mathbf{i} + c\mathbf{j} + d\mathbf{k}, \quad a,b,c,d \in \mathbb{C},
\end{equation}
tedy $\Phi$ má 8 reálných stupňů volnosti.

\section{Algebraická struktura}
Komplexní bikvaterniony tvoří algebra nekomutativních prvků s následujícími relacemi:
\begin{align*}
\mathbf{i}^2 = \mathbf{j}^2 = \mathbf{k}^2 &= -1,\\
\mathbf{i}\mathbf{j} = \mathbf{k},\quad \mathbf{j}\mathbf{k} = \mathbf{i},\quad \mathbf{k}\mathbf{i} &= \mathbf{j}.
\end{align*}
Každý koeficient může být komplexní, což umožňuje reprezentovat nejen amplitudu a fázi, ale i orientaci v~rozšířeném vnitřním prostoru.

\section{Dopad na $\theta$-funkce}
Původní $\theta$-funkce (např. Jacobiho) jsou definovány jako analytické funkce komplexní proměnné $\tau \in \mathbb{C}$. V našem případě se $\tau$ stává prvkem $\mathbb{B}$, což znamená, že $\theta$-funkce musí být rozšířeny na funkce nad $\mathbb{B}$, případně vektorové nebo spinorové zobecnění.

Zde vznikají nové otázky:
\begin{itemize}
    \item Lze definovat analytické nebo harmonické funkce nad $\mathbb{B}$?
    \item Jaký smysl má spektrální rozklad v takovém prostoru?
\end{itemize}

\section{Otázka ortogonality}
Ve stávajícím frameworku nejsou theta složky ortogonální. To komplikuje interpretaci regresních vah a zvyšuje riziko kolinearity.

V prostoru $\mathbb{B}$ lze hledat ortonormální bázi pomocí vhodné metriky (např. Hermitovské nebo kvaternionické vnitřní součiny). Jednou z možností je využít \textbf{Grand-Smithovu ortogonalizaci}, která konstruuje ortonormální bázi v nekomutativních prostorech.

\section{Další kroky}
\begin{enumerate}
    \item Zformulovat zobecněnou $\theta$-funkci na prostoru $\mathbb{B}$.
    \item Provést Grand-Smith ortonormalizaci stávajících theta složek.
    \item Vyhodnotit prediktivitu a stabilitu nové báze.
\end{enumerate}

\section*{Závěr}
Přechod k bikvaternionovému času $\tau = t + \Phi$ umožňuje modelovat skryté symetrie trhu bohatším způsobem. Ortogonální báze (např. pomocí Grand-Smithovy metody) by mohla zlepšit stabilitu modelů a redukovat kolinearitu. Tato formulace vytváří nový prostor pro experimentální validaci UBT v oblasti predikce trhů.

\end{document}
