\documentclass{article}
\usepackage[utf8]{inputenc}
\usepackage{amsmath}
\title{Predikce v Bikvaternionovém Čase a Theta Rozkladech}
\author{Výzkumná Poznámka}
\date{\today}

\begin{document}

\maketitle

\section{Motivace}

Ve stávajícím přístupu k predikci tržních časových řad se ukázalo, že rozklad pomocí theta funkcí přináší zajímavé výsledky. Nicméně přímá extrapolace těchto složek selhává, zatímco regrese (např. Ridge) přináší vysokou prediktivitu. V rámci Unified Biquaternion Theory (UBT) navrhujeme přesnější formulaci komplexního času a jeho strukturálního rozkladu.

\textbf{Poznámka}: UBT je fyzikální teorie (viz \texttt{papers/ubt\_core\_geometry/}), kde obecná relativita vzniká jako reálná projekce fundamentální bikvaternionové geometrie. Zde aplikujeme matematické struktury UBT na trhy, které \textbf{nejsou} fyzikální systémy.

\section{Bikvaternionový Čas}

Zavádíme rozšířený pojem času:
\[
\tau = t + \phi_1 \mathbf{i} + \phi_2 \mathbf{j} + \phi_3 \mathbf{k}
\]
kde reálný čas $t$ tvoří skalární složku a fáze $\phi_1, \phi_2, \phi_3$ představují bikvaternionový fázový prostor. Tento model vychází z hypotézy, že struktura trhu není pouze funkcí skalárního času, ale i latentních kvaternionových komponent.

\section{Bikvaternionový Rozklad}

Namísto standardní (neortogonální) theta báze navrhujeme použít ortonormální bázi definovanou na kvaternionovém časovém prostoru. Tato báze může být odvozena numericky nebo konstrukčně (např. Grand--Smith ortonormalizací).

\section{Predikční Mechanismy}

\subsection{Extrapolace}

Založena na pokračování fázové evoluce každé theta složky. Vhodná v případě, že trh vykazuje stabilní periodickou strukturu. Extrapolace může využívat Taylorův rozvoj, fázový drift nebo jinou aproximaci.

\subsection{Filtrace a Regrese}

Cílem je určit, které složky nesou informaci o budoucnosti. Použitelné přístupy: Ridge regrese, Lasso, Elastic Net, Kalmanův filtr, nebo filtry stavového prostoru. Tyto metody fungují jako selektivní projekce latentních struktur.

\section{Závěr}

Spojením ortonormální bikvaternionové báze a pokročilého filtračního mechanismu vzniká predikční model schopný extrahovat hlubší tržní symetrie. Experimentální ověření bude probíhat s použitím jak extrapolace, tak Ridge a Kalman filtrace.

\end{document}

