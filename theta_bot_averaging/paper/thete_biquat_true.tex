\section{Extension to Biquaternionic Time and Spatial Localization}

In the general framework of the Unified Biquaternion Theory (UBT), the fundamental field
\(\Theta(q,\tau)\) was originally defined on a complex manifold \(\mathbb{C}^5\), where
\(\tau = t + i\psi\) represents complex time, and \(q=(q_1,q_2,q_3)\) are spatial coordinates.
However, a more symmetric and conceptually complete formulation is obtained by extending both
space and time into the full biquaternionic domain.

\subsection{Biquaternionic time}

Let us generalize the time coordinate into a full biquaternion:
\[
\mathbb{T} = t_0 + i\,\mathbf{t}, \quad 
\mathbf{t} = t_1\mathbf{i} + t_2\mathbf{j} + t_3\mathbf{k}.
\]
Here \(t_0\) denotes the observable (physical) flow of time, while the imaginary
three–vector \(\mathbf{t}\) encodes internal temporal orientations associated with
conscious or informational degrees of freedom.  
The three components of \(\mathbf{t}\) can be interpreted as:
\begin{align*}
t_1 &: \text{perceptual time (immediate experience)}, \\
t_2 &: \text{memory or backward temporal projection}, \\
t_3 &: \text{anticipatory or forward temporal projection}.
\end{align*}
Thus, the imaginary quaternionic part of time forms a trinity of internal phases

