\documentclass[12pt,a4paper]{article}

% Packages
\usepackage[utf8]{inputenc}
\usepackage[T1]{fontenc}
\usepackage{amsmath,amssymb,amsthm}
\usepackage{physics}
\usepackage{graphicx}
\usepackage{hyperref}
\usepackage{geometry}

% Page setup
\geometry{margin=1in}
\setlength{\parindent}{0pt}
\setlength{\parskip}{0.5em}

% Title
\title{\textbf{Unified Biquaternion Theory: Core Geometry Refactored}\\
\large Biquaternion Spacetime Geometry with General Relativity as Real Projection}
\author{}
\date{\today}

\begin{document}

\maketitle

\begin{abstract}
We present a fundamental reformulation of Unified Biquaternion Theory (UBT) in which the spacetime geometry is intrinsically biquaternionic. General Relativity emerges not as the foundation but as the real, commutative projection of a richer non-commutative geometric structure. The metric, connection, curvature, and stress-energy tensors are all defined as biquaternion-valued objects. Classical 4D geometry and Einstein's field equations arise by taking the real part of these fundamental biquaternion structures. This formulation permits physically consistent solutions with non-vanishing imaginary components, corresponding to exotic regimes including pseudo-antigravitational behavior, phase invisibility, and local temporal drift—phenomena unobservable in standard GR but inherent to the full biquaternion geometry.
\end{abstract}

\section{Foundational Principles}

\subsection{Prohibition of Classical Metric as Fundamental}

\textbf{Fundamental Axiom}: The classical real metric tensor $g_{\mu\nu}$ is \textbf{NOT} a fundamental object. It is a derived, limiting projection.

All prior formulations that postulate $g_{\mu\nu}$ as the starting point are \textbf{explicitly rejected}. Instead, we demand:

\begin{equation}
g_{\mu\nu} := \mathrm{Re}(\mathcal{G}_{\mu\nu})
\end{equation}

where $\mathcal{G}_{\mu\nu}$ is the fundamental \textbf{biquaternion metric}.

\textbf{Mandatory Rule}: No use of $g_{\mu\nu}$ is permitted without explicit reference to its origin as the real projection of $\mathcal{G}_{\mu\nu}$.

\subsection{Biquaternion Algebra}

The algebra of biquaternions $\mathbb{B}$ consists of quaternions with complex coefficients. A general biquaternion is written:

\begin{equation}
q = q_0 + q_1 \mathbf{i} + q_2 \mathbf{j} + q_3 \mathbf{k}, \quad q_\alpha \in \mathbb{C}
\end{equation}

where $\mathbf{i}, \mathbf{j}, \mathbf{k}$ are the quaternion basis elements satisfying Hamilton's relations:

\begin{align}
\mathbf{i}^2 = \mathbf{j}^2 = \mathbf{k}^2 &= -1 \\
\mathbf{i}\mathbf{j} = \mathbf{k}, \quad \mathbf{j}\mathbf{k} &= \mathbf{i}, \quad \mathbf{k}\mathbf{i} = \mathbf{j} \\
\mathbf{j}\mathbf{i} = -\mathbf{k}, \quad \mathbf{k}\mathbf{j} &= -\mathbf{i}, \quad \mathbf{i}\mathbf{k} = -\mathbf{j}
\end{align}

Each biquaternion has 8 real degrees of freedom (4 complex coefficients).

\textbf{Critical Property}: Biquaternion multiplication is \textbf{non-commutative}. We \textbf{never} assume commutativity of components.

\subsection{Decomposition Structure}

A biquaternion $q \in \mathbb{B}$ can be decomposed as:

\begin{equation}
q = (a + I b) + \mathbf{J} \cdot \mathbf{v}
\end{equation}

where:
\begin{itemize}
\item $a, b \in \mathbb{R}$ are real scalars
\item $I = \sqrt{-1}$ is the complex imaginary unit
\item $\mathbf{J} = (\mathbf{i}, \mathbf{j}, \mathbf{k})$ is the quaternion basis vector
\item $\mathbf{v} \in \mathbb{C}^3$ is a complex 3-vector
\end{itemize}

Alternatively, written in full:

\begin{equation}
q = s + I h + \mathbf{J} \cdot \mathbf{k}
\end{equation}

where $s, h \in \mathbb{R}$ and $\mathbf{k} = (k_1, k_2, k_3)$ with $k_i \in \mathbb{C}$.

\section{Fundamental Biquaternion Metric}

\subsection{Definition of the Metric Tensor}

The fundamental geometric object of spacetime is the \textbf{biquaternion metric}:

\begin{equation}
\mathcal{G}_{\mu\nu}(x) \in \mathbb{B}
\end{equation}

defined at each spacetime point $x^\mu$.

\subsection{Decomposition of the Metric}

The biquaternion metric admits the general decomposition:

\begin{equation}
\mathcal{G}_{\mu\nu} = g_{\mu\nu} + I h_{\mu\nu} + \mathbf{J} \cdot \mathbf{k}_{\mu\nu}
\end{equation}

where:
\begin{itemize}
\item $g_{\mu\nu}(x) \in \mathbb{R}$ is the \textbf{real projection} (GR sector)
\item $h_{\mu\nu}(x) \in \mathbb{R}$ is the \textbf{phase geometry}
\item $\mathbf{k}_{\mu\nu}(x) = (k_{\mu\nu}^1, k_{\mu\nu}^2, k_{\mu\nu}^3)$ with $k_{\mu\nu}^a \in \mathbb{C}$ represents \textbf{inertial and causal geometry}
\end{itemize}

\textbf{Physical Interpretation}:
\begin{itemize}
\item $g_{\mu\nu}$: Observable spacetime metric (classical GR)
\item $h_{\mu\nu}$: Phase structure (non-observable in GR regime)
\item $\mathbf{k}_{\mu\nu}$: Extended causal and inertial structure
\end{itemize}

\subsection{Hermiticity and Reality Conditions}

For the metric to be physically meaningful, we require:

\begin{equation}
\mathcal{G}_{\mu\nu}^\dagger = \mathcal{G}_{\nu\mu}
\end{equation}

where $\dagger$ denotes biquaternion conjugation (both complex and quaternion conjugation).

This ensures that $g_{\mu\nu}$ is real and symmetric.

\section{Biquaternion Tetrad Formalism}

\subsection{Mandatory Tetrad Structure}

The metric \textbf{must not} be introduced directly. Instead, we define the fundamental \textbf{biquaternion tetrad field}:

\begin{equation}
E_\mu(x) \in \mathbb{B}
\end{equation}

representing the local frame at each point.

\subsection{Metric from Tetrad}

The metric is defined \textbf{exclusively} through the tetrad:

\begin{equation}
\mathcal{G}_{\mu\nu} := \mathrm{Sc}(E_\mu E_\nu^\dagger)
\end{equation}

where $\mathrm{Sc}$ denotes the scalar part of the biquaternion product.\footnote{For a biquaternion $q = q_0 + q_1 \mathbf{i} + q_2 \mathbf{j} + q_3 \mathbf{k}$ with $q_\alpha \in \mathbb{C}$, we have $\mathrm{Sc}(q) = q_0$. For a product of biquaternions, $\mathrm{Sc}(AB)$ extracts only the $q_0$ component of the result.}

Equivalently, using the symmetrized biquaternion product:

\begin{equation}
\mathcal{G}_{\mu\nu} = \frac{1}{2}\left(E_\mu E_\nu^\dagger + E_\nu E_\mu^\dagger\right)
\end{equation}

\textbf{Prohibition}: Direct introduction of $g_{\mu\nu}$ or ad-hoc projections without tetrad construction are \textbf{forbidden}.

\subsection{Tetrad Decomposition}

Each tetrad component can be written:

\begin{equation}
E_\mu = e_\mu + I f_\mu + \mathbf{J} \cdot \boldsymbol{\xi}_\mu
\end{equation}

where $e_\mu \in \mathbb{R}$, $f_\mu \in \mathbb{R}$, and $\boldsymbol{\xi}_\mu \in \mathbb{C}^3$.

\section{Biquaternion Connection}

\subsection{Replacement of Christoffel Symbols}

Christoffel symbols $\Gamma^\lambda_{\mu\nu}$ are \textbf{NOT fundamental}. They are artifacts of the real projection.

The fundamental object is the \textbf{biquaternion connection}:

\begin{equation}
\Omega_\mu(x) \in \mathbb{B}
\end{equation}

\subsection{Covariant Derivative of Tetrad}

The biquaternion connection is defined through the torsion-free compatibility condition with the tetrad:

\begin{equation}
\nabla_\mu E_\nu = \partial_\mu E_\nu + \Omega_\mu \circ E_\nu = 0
\end{equation}

where $\circ$ denotes biquaternion multiplication (non-commutative).

\textbf{Important}: We do \textbf{not} simplify commutators or assume associativity holds trivially. The full non-commutative structure must be preserved.

\textbf{Note on Christoffel Symbols}: In the real projection, the connection $\Omega_\mu$ induces effective Christoffel symbols $\Gamma^\lambda_{\mu\nu}$ via:

\begin{equation}
\Gamma^\lambda_{\mu\nu} = \mathrm{Re}\left(\partial_\mu e_\nu^\lambda + \omega_\mu \circ e_\nu^\lambda\right)
\end{equation}

where $e_\nu^\lambda$ are the real tetrad components. However, $\Gamma^\lambda_{\mu\nu}$ are \textbf{derived} quantities, not fundamental. The fundamental object is always $\Omega_\mu$.

\subsection{Connection Decomposition}

The biquaternion connection decomposes as:

\begin{equation}
\Omega_\mu = \omega_\mu + I \alpha_\mu + \mathbf{J} \cdot \mathbf{A}_\mu
\end{equation}

where:
\begin{itemize}
\item $\omega_\mu \in \mathbb{R}$: spin connection (real part)
\item $\alpha_\mu \in \mathbb{R}$: phase connection
\item $\mathbf{A}_\mu \in \mathbb{C}^3$: gauge-like connection (complex 3-vector)
\end{itemize}

\section{Biquaternion Curvature}

\subsection{Curvature Tensor Definition}

The biquaternion curvature is defined via the connection:

\begin{equation}
\mathcal{R}_{\mu\nu} = \partial_\mu \Omega_\nu - \partial_\nu \Omega_\mu + [\Omega_\mu, \Omega_\nu]
\end{equation}

where $[\cdot, \cdot]$ denotes the biquaternion commutator:

\begin{equation}
[\Omega_\mu, \Omega_\nu] = \Omega_\mu \Omega_\nu - \Omega_\nu \Omega_\mu
\end{equation}

Due to non-commutativity, this term is generically \textbf{non-zero}.

\subsection{Ricci Tensor}

The biquaternion Ricci tensor is constructed as:

\begin{equation}
\mathcal{R}_{\nu\sigma} = E^\mu \mathcal{R}_{\mu\nu} E_\sigma
\end{equation}

where contraction is performed using the biquaternion tetrad.

\subsection{Real Projection to Classical Ricci Tensor}

Only after defining the biquaternion Ricci tensor do we permit:

\begin{equation}
R_{\mu\nu} := \mathrm{Re}(\mathcal{R}_{\mu\nu})
\end{equation}

This is the classical Ricci tensor of General Relativity.

\section{Biquaternion Stress-Energy Tensor}

\subsection{Prohibition of Classical Definition}

Classical definitions of the real stress-energy tensor $T_{\mu\nu}$ as fundamental are \textbf{abolished}.

\subsection{Fundamental Biquaternion Stress-Energy}

The fundamental object is the \textbf{biquaternion stress-energy tensor}:

\begin{equation}
\mathcal{T}_{\mu\nu} = \langle D_\mu \Theta, D_\nu \Theta \rangle_\mathbb{B} - \frac{1}{2} \mathcal{G}_{\mu\nu} \langle D\Theta, D\Theta \rangle
\end{equation}

where:
\begin{itemize}
\item $\Theta(x) \in \mathbb{B}$ is the fundamental biquaternion field
\item $D_\mu \Theta = \partial_\mu \Theta + \Omega_\mu \Theta$ is the covariant derivative
\item $\langle \cdot, \cdot \rangle_\mathbb{B}$ is the biquaternion inner product
\end{itemize}

\subsection{Biquaternion Inner Product}

The inner product on biquaternions is defined as:

\begin{equation}
\langle A, B \rangle_\mathbb{B} = \mathrm{Sc}(A B^\dagger)
\end{equation}

where $B^\dagger$ is the biquaternion conjugate.

\subsection{Real Projection}

The observable stress-energy tensor is:

\begin{equation}
T_{\mu\nu} := \mathrm{Re}(\mathcal{T}_{\mu\nu})
\end{equation}

This is the only stress-energy tensor accessible to classical measurements.

\section{Field Equations}

\subsection{Prohibition of Einstein Field Equations as Fundamental}

The classical Einstein field equations:

\begin{equation}
G_{\mu\nu} = \kappa T_{\mu\nu}
\end{equation}

are \textbf{NOT} fundamental. They are \textbf{forbidden} as the starting point.

\subsection{Fundamental Biquaternion Field Equation}

The fundamental field equation of UBT is:

\begin{equation}
\mathcal{G}_{\mu\nu} = \kappa \mathcal{T}_{\mu\nu}
\end{equation}

where both sides are biquaternion-valued.

\textbf{Explicit Statement}: The real projection yields Einstein's equations:

\begin{equation}
\mathrm{Re}(\mathcal{G}_{\mu\nu}) = \kappa \mathrm{Re}(\mathcal{T}_{\mu\nu}) \quad \Rightarrow \quad G_{\mu\nu} = \kappa T_{\mu\nu}
\end{equation}

Thus, \textbf{General Relativity emerges as the real, commutative projection of the fundamental biquaternion geometry}.

\subsection{Imaginary Sector Equations}

The imaginary components satisfy:

\begin{align}
\mathrm{Im}(\mathcal{G}_{\mu\nu}) &= \kappa \mathrm{Im}(\mathcal{T}_{\mu\nu}) \\
\mathbf{J} \cdot \mathrm{Vec}(\mathcal{G}_{\mu\nu}) &= \kappa \mathbf{J} \cdot \mathrm{Vec}(\mathcal{T}_{\mu\nu})
\end{align}

These equations govern the phase, inertial, and causal sectors invisible to classical GR.

\section{Exotic Regimes and Unobservable Sectors}

\subsection{Non-Vanishing Imaginary Metric}

Solutions with:

\begin{equation}
\mathrm{Im}(\mathcal{G}_{\mu\nu}) \neq 0
\end{equation}

are \textbf{physically consistent within UBT} but \textbf{not observable in standard GR}.

\subsection{Exotic Phenomena}

Non-zero imaginary components correspond to:

\begin{itemize}
\item \textbf{Pseudo-antigravitational behavior}: Effective repulsion in the phase sector
\item \textbf{Phase invisibility}: Matter configurations that couple only to $\mathrm{Im}(\mathcal{G}_{\mu\nu})$ and are invisible to classical observers
\item \textbf{Local temporal drift}: Variations in the flow of time not captured by $g_{00}$ alone, encoded in $h_{00}$ and $k_{00}^a$
\item \textbf{Causal structure modification}: Lightcones extended or modified by quaternion components
\end{itemize}

\subsection{Observational Constraints}

For consistency with observations:

\begin{itemize}
\item In the solar system and cosmological observations, $\mathrm{Im}(\mathcal{G}_{\mu\nu}) \ll \mathrm{Re}(\mathcal{G}_{\mu\nu})$
\item Dark matter and dark energy effects may correspond to \textbf{non-zero but small} imaginary sectors
\item Quantum gravitational regimes near Planck scale may exhibit $\mathrm{Im}(\mathcal{G}_{\mu\nu}) \sim \mathrm{Re}(\mathcal{G}_{\mu\nu})$
\end{itemize}

\section{Meta-Commentary: GR as Emergent Limit}

\subsection{Foundational Statement}

\textbf{General Relativity arises as the real, commutative projection of the fundamental biquaternion geometry of spacetime.}

Apparent violations of GR, such as:
\begin{itemize}
\item Antigravitation without exotic matter
\item Causal drift without coordinate singularities
\item Phase-shifted matter fields
\end{itemize}

correspond to \textbf{non-real sectors of the metric and curvature}, not to violations of energy conditions or introduction of exotic matter.

\subsection{Classical vs. Fundamental Reality}

The observable universe corresponds to $\mathrm{Re}(\mathcal{G}_{\mu\nu})$ due to:
\begin{itemize}
\item Standard matter fields coupling predominantly to real metric components
\item Measurement apparatuses themselves described by real projections
\item Decoherence suppressing access to phase sectors
\end{itemize}

However, the \textbf{fundamental reality} is biquaternionic. GR is a \textbf{shadow} of the complete geometry.

\subsection{No Violation of Energy Conservation}

The full biquaternion stress-energy tensor satisfies:

\begin{equation}
\nabla_\mu \mathcal{T}^{\mu\nu} = 0
\end{equation}

in the biquaternion sense. Taking the real part:

\begin{equation}
\nabla_\mu T^{\mu\nu} = 0
\end{equation}

recovers classical energy-momentum conservation.

\textbf{Critical Point}: Energy is \textbf{not} created ex nihilo. Apparent energy violations in the real sector are compensated by flows in the imaginary sector.

\section{Mandatory Prohibitions}

\subsection{What is Forbidden}

The following are \textbf{strictly prohibited} in UBT:

\begin{enumerate}
\item \textbf{Using GR as an axiom}: The metric $g_{\mu\nu}$ cannot be postulated directly.

\item \textbf{Simplifying biquaternions to complex numbers}: The full 8-component structure must be preserved.

\item \textbf{Breaking global causality}: While local causal structure can be extended, acausal loops are forbidden.

\item \textbf{Identifying observable world with fundamental reality}: What we observe is $\mathrm{Re}(\mathcal{G}_{\mu\nu})$, not the full $\mathcal{G}_{\mu\nu}$.

\item \textbf{Introducing energy ex nihilo}: Total biquaternion energy-momentum is conserved.
\end{enumerate}

\subsection{Methodological Requirements}

All formulations must:
\begin{itemize}
\item Start with biquaternion objects
\item Derive real projections explicitly
\item Preserve non-commutativity throughout
\item Specify how classical limits are taken
\end{itemize}

\section{Conclusion}

We have established a complete refactoring of Unified Biquaternion Theory where:

\begin{enumerate}
\item The fundamental spacetime geometry is \textbf{biquaternionic}
\item Metric, connection, curvature, and stress-energy are all $\mathbb{B}$-valued
\item General Relativity emerges as $\mathrm{Re}(\mathcal{G}_{\mu\nu})$
\item Exotic phenomena correspond to $\mathrm{Im}(\mathcal{G}_{\mu\nu}) \neq 0$
\item No violation of energy conservation or causality occurs
\end{enumerate}

\textbf{UBT possesses a closed biquaternion geometry from which GR emerges solely as the limiting real sector.}

\vspace{1em}

\noindent\textbf{Declaration}: This document establishes the mathematical foundation. Physical predictions, experimental tests, and phenomenological consequences require separate detailed analysis.

\end{document}
