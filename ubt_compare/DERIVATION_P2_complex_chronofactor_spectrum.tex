% DERIVATION_P2_complex_chronofactor_spectrum.tex
% LaTeX-ready section for inclusion in papers or appendices.
% Requires: amsmath, amssymb (standard)

\section{Symbolic Derivation: Complex Chronofactor $\tau = t + i\psi$\\
         Spectral Consequences and Links to $S_\Theta$ and $\Sigma_\Theta$}

\subsection{Setup and Assumptions}

\paragraph{Chronofactor.}
Let the \emph{chronofactor} be
\begin{equation}
  \tau = t + i\psi, \qquad t, \psi \in \mathbb{R},
\end{equation}
where $t$ is physical (real) time and $\psi$ is the imaginary displacement.

\paragraph{Matrix field.}
Let $\Theta(\tau)$ be an $n\times n$ complex matrix field, holomorphic in a strip $|\psi|<\psi_{\max}$.
Define the \emph{overlap matrix}
\begin{equation}
  O(\tau) = \Theta^\dagger(\tau)\,\Theta(\tau),
\end{equation}
which is positive semidefinite on the real-time slice $\tau = t$.

\paragraph{UBT invariants.}
\begin{align}
  S_\Theta  &:= k_B \log\det(\Theta^\dagger\Theta), \\
  \Sigma_\Theta &:= k_B \arg\det\Theta = k_B\,\mathrm{Im}\log\det\Theta.
\end{align}

\paragraph{Generator decomposition.}
Let $G = A + iB$ with $A = \tfrac{1}{2}(G+G^\dagger)$ (Hermitian part) and
$B = \tfrac{1}{2i}(G-G^\dagger)$ (anti-Hermitian part, $B$ Hermitian).

% -----------------------------------------------------------------------
\subsection{Linear Complex-$\tau$ Flow}

\begin{equation}
  \partial_\tau \Theta(\tau) = G\,\Theta(\tau), \qquad \Theta(0) = \Theta_0.
\end{equation}

\begin{proposition}[Exact solution]\label{prop:exact}
  The unique holomorphic solution is
  \begin{equation}
    \Theta(\tau) = e^{\tau G}\,\Theta_0.
  \end{equation}
\end{proposition}

\begin{proof}
  Differentiate $e^{\tau G}\Theta_0$ with respect to $\tau$; uniqueness from
  Picard--Lindel\"of on the linear system.
\end{proof}

\paragraph{Spectral decomposition.}
Assume $G$ diagonalizable: $G = V\,\mathrm{diag}(\mu_1,\ldots,\mu_n)\,V^{-1}$
with $\mu_k = \alpha_k + i\omega_k \in \mathbb{C}$.  Then
\begin{equation}
  e^{\tau G} = V\,\mathrm{diag}(e^{\tau\mu_1},\ldots,e^{\tau\mu_n})\,V^{-1}.
\end{equation}

% -----------------------------------------------------------------------
\subsection{What Changes When $\tau$ is Complex}

Setting $\tau = t + i\psi$, $\mu_k = \alpha_k + i\omega_k$:
\begin{equation}\label{eq:mode-factor}
  \boxed{e^{\tau\mu_k} = e^{t\alpha_k - \psi\omega_k}\cdot e^{i(t\omega_k + \psi\alpha_k)}.}
\end{equation}
\begin{itemize}
  \item \textbf{Magnitude:} $|e^{\tau\mu_k}|=e^{t\alpha_k-\psi\omega_k}$.
        Real time $t$ drives growth/decay; imaginary displacement $\psi$ modulates
        amplitude through the imaginary eigenvalue part $\omega_k$.
  \item \textbf{Phase:} $\arg e^{\tau\mu_k}=t\omega_k+\psi\alpha_k$.
        Complex $\tau$ \emph{couples} the oscillation and growth/decay channels.
\end{itemize}

\paragraph{Commuting vs.\ non-commuting generators.}
With $G=A+iB$:
\begin{equation}
  e^{\tau G} = e^{(tA-\psi B)+i(tB+\psi A)}.
\end{equation}
\textbf{Commuting case} ($[A,B]=0$):
\begin{equation}
  e^{\tau G} = e^{tA-\psi B}\cdot e^{i(tB+\psi A)}.
\end{equation}
\textbf{General case} (BCH first-order correction):
\begin{equation}\label{eq:BCH}
  e^{\tau G} = e^{(tA-\psi B)+i(tB+\psi A)}\cdot
               \exp\!\Bigl(\tfrac{i}{2}|\tau|^2[A,B] + O\bigl([A,B]^2\bigr)\Bigr),
               \quad |\tau|^2 = t^2+\psi^2.
\end{equation}
The correction grows with $|\tau|^2$ for non-commuting generators.

% -----------------------------------------------------------------------
\subsection{Consequences for $O(\tau)=\Theta^\dagger\Theta$}

Differentiating $O=\Theta^\dagger\Theta$ with respect to $t$ (fixed $\psi$):
\begin{equation}\label{eq:Odot}
  \boxed{\partial_t O = 2\,\Theta^\dagger\,\mathrm{Herm}(G)\,\Theta = 2\,\Theta^\dagger A\,\Theta.}
\end{equation}
\begin{corollary}
  If $G$ is anti-Hermitian ($A=0$), then $\partial_t O=0$: eigenvalues of $O$
  are conserved; only the phase of $\Theta$ evolves.
\end{corollary}

When $\psi=\psi(t)$ is time-dependent, the effective generator is
$\tilde G=(1+i\psi')G$ and
\begin{equation}\label{eq:Odot-psi}
  \frac{d}{dt}O\Big|_{\psi=\psi(t)} = 2\,\Theta^\dagger\bigl(A - \psi'(t)B\bigr)\Theta.
\end{equation}
Even for $A=0$, a non-zero $\psi'(t)$ induces eigenvalue drift via the
anti-Hermitian part $B$, causing \emph{spectral broadening}.

% -----------------------------------------------------------------------
\subsection{Determinant Channel: $\log\det\Theta$ and $\log\det(\Theta^\dagger\Theta)$}

\begin{proposition}[log det identity]\label{prop:logdet}
  Under the linear flow $\Theta(\tau)=e^{\tau G}\Theta_0$:
  \begin{equation}
    \boxed{\log\det\Theta(\tau) = \tau\,\mathrm{Tr}\,G + \log\det\Theta_0.}
  \end{equation}
\end{proposition}
\begin{proof}
  $\det(e^{\tau G})=e^{\tau\,\mathrm{Tr}\,G}$ by Jacobi's formula. \qed
\end{proof}

With $\tau=t+i\psi$ and $\mathrm{Tr}\,G = \mathrm{Tr}\,A + i\,\mathrm{Tr}\,B$:
\begin{align}
  \mathrm{Re}\log\det\Theta &= t\,\mathrm{Tr}\,A - \psi\,\mathrm{Tr}\,B + \mathrm{Re}\log\det\Theta_0,
  \label{eq:relogdet}\\
  \mathrm{Im}\log\det\Theta &= t\,\mathrm{Tr}\,B + \psi\,\mathrm{Tr}\,A + \mathrm{Im}\log\det\Theta_0.
  \label{eq:imlogdet}
\end{align}

\paragraph{The $2\times\mathrm{Re}$ relationship.}
On any slice where $\Theta^\dagger$ is the conjugate transpose at the same $\tau$:
\begin{equation}
  \log\det(\Theta^\dagger\Theta) = \overline{\log\det\Theta} + \log\det\Theta
                                 = 2\,\mathrm{Re}\log\det\Theta.
\end{equation}

% -----------------------------------------------------------------------
\subsection{Phase Channel: $\arg\det\Theta$ as Holonomy}

\begin{proposition}[Explicit $\Sigma_\Theta$]\label{prop:sigma}
  \begin{equation}
    \boxed{\Sigma_\Theta(\tau) = k_B\bigl(t\,\mathrm{Tr}\,B + \psi\,\mathrm{Tr}\,A
                                  + \mathrm{Im}\log\det\Theta_0\bigr).}
  \end{equation}
\end{proposition}

\begin{proposition}[Explicit $S_\Theta$]\label{prop:entropy}
  \begin{equation}
    \boxed{S_\Theta(\tau) = 2k_B\bigl(t\,\mathrm{Tr}\,A - \psi\,\mathrm{Tr}\,B
                             + \mathrm{Re}\log\det\Theta_0\bigr).}
  \end{equation}
\end{proposition}

When $\psi$ traces a closed loop (returning to its initial value) and no zero
of $\det\Theta$ is enclosed, the net change in $\Sigma_\Theta$ vanishes.
Encircling a zero contributes $2\pi k_B$ per winding (monodromy).

% -----------------------------------------------------------------------
\subsection{Non-Hermitian Case: Pseudospectrum}

For non-Hermitian $G$, right and left eigenvectors differ:
\begin{equation}
  G v_k = \mu_k v_k, \quad w_k^\dagger G = \mu_k w_k^\dagger, \quad
  w_j^\dagger v_k = \delta_{jk}.
\end{equation}

\begin{definition}[$\epsilon$-pseudospectrum]
  \(\Lambda_\epsilon(G) = \{z\in\mathbb{C} : \|(G-zI)^{-1}\|\geq\epsilon^{-1}\}\).
\end{definition}

Even when all $\mathrm{Re}\,\mu_k < 0$, a non-normal $G$ can produce transient growth
of $\|e^{\tau G}\|$ for finite $t$.  This appears as non-monotone eigenvalue
trajectories in $O(\tau)$ --- a signature not predictable from eigenvalues alone.

% -----------------------------------------------------------------------
\subsection{Diagnostic Invariants (Discriminators D1--D3)}

\paragraph{D1: Phase--Entropy Coupling.}
\begin{equation}\label{eq:D1}
  \boxed{C_{12}(t) = \frac{\partial_t\Sigma_\Theta}{\partial_t S_\Theta}.}
\end{equation}
Under real $\tau$: $C_{12}=\mathrm{Tr}\,B/(2\,\mathrm{Tr}\,A)$ (constant).\\
Under complex $\tau$ with $\psi=\psi(t)$:
\begin{equation}
  C_{12}(t) = \frac{\mathrm{Tr}\,B + \psi'(t)\,\mathrm{Tr}\,A}
                   {2\bigl(\mathrm{Tr}\,A - \psi'(t)\,\mathrm{Tr}\,B\bigr)},
\end{equation}
which is time-varying whenever $\psi'(t)\neq 0$.

\paragraph{D2: Spectral Conservation Test.}
\begin{equation}
  \lambda_i(O(t)) = \mathrm{const} \iff \mathrm{Herm}(G_{\mathrm{eff}}) = 0.
\end{equation}
Under complex $\tau$ with $A=0$, the effective Hermitian part is $-\psi'B\neq 0$
when $\psi'\neq 0$, producing observable eigenvalue drift.

\paragraph{D3: Mode Pairing and Oscillatory $\Sigma_\Theta$.}
\begin{equation}
  \text{Oscillatory }\Sigma_\Theta(t)\text{ with monotone }S_\Theta(t)
  \iff \mathrm{Tr}\,B \neq 0 \text{ and complex-conjugate eigenpairs.}
\end{equation}
The dominant frequency in $\Sigma_\Theta$ is $f_0 = \mathrm{Tr}\,B/(2\pi k_B)$.

% -----------------------------------------------------------------------
\subsection*{Appendix A: $2\times 2$ Worked Example}

Let $G=\mathrm{diag}(a+ib,\,c+id)$, $\Theta_0=I_2$.  Then:
\begin{align}
  \Theta(\tau) &= \mathrm{diag}\!\bigl(e^{\tau(a+ib)},\,e^{\tau(c+id)}\bigr),\\
  \log\det\Theta &= \tau\bigl((a+c)+i(b+d)\bigr),\\
  S_\Theta &= 2k_B\bigl(t(a+c)-\psi(b+d)\bigr),\\
  \Sigma_\Theta &= k_B\bigl(t(b+d)+\psi(a+c)\bigr),\\
  O(\tau) &= \mathrm{diag}\!\bigl(e^{2(ta-\psi b)},\,e^{2(tc-\psi d)}\bigr).
\end{align}
Eigenvalue drift: $\partial_\psi\lambda_1 = -2b\lambda_1$, confirming that the
imaginary eigenvalue part $b=\mathrm{Im}\,\mu_1$ modulates amplitude under $\psi$.

% -----------------------------------------------------------------------
\subsection*{Appendix B: Heat-Kernel / Tr\,log Representation}

The identity $\log\det\Theta=\mathrm{Tr}\log\Theta$ holds on appropriate branches.
For the linear flow:
\begin{equation}
  \mathrm{Tr}\log\Theta(\tau) = \tau\,\mathrm{Tr}\,G + \mathrm{Tr}\log\Theta_0.
\end{equation}
The heat-kernel representation
$\mathrm{Tr}\log\Theta = -\int_0^\infty s^{-1}\mathrm{Tr}(e^{-s\Theta}-e^{-s})\,ds$
acquires oscillatory integrands as $\psi$ varies, generating the same oscillations
that appear in $\Sigma_\Theta$.

% -----------------------------------------------------------------------
\subsection*{Appendix C: Regularization Near $\det\Theta\to 0$}

When $\mathrm{Re}(\tau\mu_k)\to-\infty$ for some $k$, $\log\det\Theta\to-\infty$.
Standard regularizations:
\begin{enumerate}
  \item \textbf{Tikhonov:} $\log\det(\Theta+\varepsilon I)=\sum_k\log(\lambda_k+\varepsilon)$.
  \item \textbf{Zeta-function:} $\log\det\Theta = -\zeta'_\Theta(0)$, analytic continuation
        of $\zeta_\Theta(s)=\mathrm{Tr}(\Theta^{-s})$.
  \item \textbf{Conditional:} Sum only over non-zero eigenvalues.
\end{enumerate}
Physically, $S_\Theta\to-\infty$ signals collapse onto a degenerate subspace.
Near such a point the pseudospectrum expands, indicating high sensitivity.
